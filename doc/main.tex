\documentclass{report}

\usepackage{amsmath,amssymb}
\usepackage{graphicx}
\usepackage{booktabs}
%\usepackage{chemfig}

\newcommand{\pd}[2]{ \frac{\partial #1}{\partial #2} }
\newcommand{\order}[1]{ \mathcal{O}\left( \ensuremath{#1} \right) }
\newcommand{\m}[1]{ \mathrm{#1} }

\begin{document}

\chapter{Forming OEUs}
\section{Overview}
OTUs were clustered into OEUs in three steps: filtering, agglomeration, and quality control. OEUs consist of abundant OTUs with similar abundance profiles across depth, computed using the following steps:

\begin{enumerate}
\item Remove lowly-abundant OTUs from analysis.
\item Compute a matrix of divergences between remaining OTUs. 
\item Agglomerate OTUs into clusters with unsupervised clustering.
\item Prune clusters by removing OTUs that do not meet quality thresholds.
\end{enumerate}

\section{Specific method}
We removed OTUs with fewer than 2000 reads and/or that appeared in fewer than 2 samples. Of $\mathord{\sim} 11800$ original OTUs and 6.7 million total reads, 319 OTUs and 6.0 million reads continued into the downstream analysis.

We computed the relative abundance of each OTU in every sample (i.e., normalized within each sample) and then created abundance profiles for each OTU (i.e., normalized across samples for an OTU). The abundance profile for each OTU across depths sums to unity. The divergence between two OTUs was defined as the Euclidean distance between their abundance profiles. OTUs with similar distributions across depths have low divergences regardless of their overall abundance.

We created a hierarchical clustering of OTUs using the divergence matrix via Ward's method (implemented in the {\tt hclust} command in the R statistical computing language). Clusters were formed by cutting the tree at a height that produces a user-selected number of clusters. Higher numbers of clusters produced more numerous, smaller, tighter OTUs; lower numbers produced larger clusters. We selected 50 clusters as a trade-off between cluster quality and larger, more interesting clusters.

We pruned clusters by computing the Pearson correlation between all OTUs in a cluster. If the cluster with the lowest mean correlation fell below 0.75, it was removed from the cluster. This process was repeated until the lowest mean correlation  was above 0.75 or the cluster was pruned to a single OTU.

The remaining clusters were designated as OEUs. 283 OTUs representing 5.5 million reads were included in the OEUs, representing 88\% of the original OTUs and 82\% of the total original reads.


\chapter{Biogeochemical model}
\section{Model scope}
We developed a quantiative, conceptual model of the microbiology and geochemistry in Mystic Lake's
hypolimnion to compare against the observed data. We expected that the model would confirm
our understanding of the basic mechanisms that lead to redox zonation and the distribution
of specific bacterial taxa in the hypolimnion. The model is conceptual rather than operational: it
is designed to test our understanding of general principles in community structure in
eutrophic dimictic lakes rather than predict any particular observable from Mystic Lake with
high accuracy.

We treat the hypolimnion as a mostly closed system of cycling nutrients. Compartments are
linked by transport processes, and reactions within compartments interconvert simulated chemical
species. Interaction with the environment outside the hypolimnion is represented by just two
processes: input of oxidizable carbon in the upper compartments and input of methane from
the sediment. The sediment and metalimnion are otherwise ignored.

\section{Model}
\subsection{Chemical species and reactions}
We use a simplified set of chemical species to avoid excessive parameterization and
because we need only predict values for which we have experimental comparison. Table
\ref{tab:chemical_species} lists all chemical species simulated.

In nature, carbon is added to the hypolimnion through particulate biomass, humics from
rainwater runoff, photosynthesis,
etc. There are many carbon species with differing oxidation states. In this model, we 
simulate only one species of carbon (C) that has only one oxidation state.

In the model, many chemical compounds are considered abundant or unimportant to the simulated biogeochemical cycles and are therefore excluded. Carbon dioxide is considered ubiquitous and abundant. Its production and consumption are not tracked and reactions that consume
CO$_2$ are assumed to proceed without interruption. Nitrogen gas, produced by denitrification
and iron oxidation on nitrate, is ignored.

The pH in Mystic Lake's hypolimnion varies between about 6 and 7. We neglected effects of pH on chemical species formation.

\begin{table}
\centering
\begin{tabular}{ l l l }
\toprule
designation & name  & representative compounds \\
\midrule
O   & dissolved oxygen    & O$_2$ \\
C   & oxidizable carbon & cyanobacteria biomass, glucose, acetate \\
N$^+$ & oxidized nitrogen &  nitrate, nitrite \\
N$^-$ & reduced nitrogen  & ammonia \\
Fe$^+$ &  oxidized iron  & Fe(III) compounds  \\
Fe$^-$ &  reduced iron  & Fe(II)  \\
S$^+$ &  oxidized sulfur  & sulfate compounds \\
S$^-$ &  reduced sulfur  & sulfide compounds  \\
M   &   methane &   CH$_4$ \\
\bottomrule
\end{tabular}
\caption{Chemical species included in the model.}
\label{tab:chemical_species}
\end{table}

\subsection{Inferred microbial biomass}
Typically an implicit biomass model is used when the abundance of microbes catalyzing
simulated reactions is irrelevant to the study. Explicit biomass models, in contrast, require parameterizing the growth kinetics for each modeled microbial species and so typically only include a few biological species. 

Rather than predict the observable abundance of a few chemical species, here we focus on the
distribution of bacteria performing similar metabolisms. If each organism performed only
one metabolism as defined by the reactions in the model, all organisms performing that
metabolism would group into an ecological guild. We assume that the
rate of a reaction at a given depth is proportional to the number of individuals in the guild with the
corresponding metabolism in that spatial compartment:
\begin{equation}
  (\text{reaction rate}) = (\text{rate per individual}) \times (\text{number of guild members}).
\end{equation}
Even if the constant of proportionality (rate per individual) is not known, the distribution of
that guild's biomass across
depth can be inferred from the relative reaction rates at each depth.

As noted in Hunter et al.\cite{hunterkinetic1998}, an implicit biomass model is equivalent to assuming that population
biomass is only limited by the availability of the energy source, that is, the guild
quickly equilibrates to the size of the available niche.

\section{Mechanics: Transport and reactions}
The rate of change in the concentration of a chemical species $X$ at a depth $i$ is
\begin{equation}
  \frac{\partial X_i}{\partial t} = \left( \text{transport terms} \right) + \left(
  \text{reaction terms} \right) + \left(\text{source terms}\right),
\end{equation}
where low $i$ refers to low depth in meters, i.e., vertically higher in the water column.
The simulation proceeds in $N$ compartments, which we spaced at one meter to be comparable
to the collected chemical and biological data. The initial concentrations are set and the
simulation proceeds for a time $T$, during which the chemical species concentrations and
reaction rates are recorded. This time roughly corresponds to the period between the
movement of the thermocline up the water column in spring and the breakdown of
stratification in fall.

\subsection{Transport: Diffusion and precipitation}
Most chemical species are treated as dissolved in the water column. In the time and length scales relevant to the hypolimnion ecosystem, molecular diffusion is slow compared to bulk transport processes like vertical eddy diffusion. To model these bulk transport processes, most chemical species are transported by simple diffusion with rate $D \left(X_{i-1} - X_i\right) + D \left(X_{i+1} - X_i \right)$, where the diffusion constant $D$ is the same for all chemical species, since it represents a bulk transport process. To account for the boundaries at the metalimnion and sediment, the first term is excluded in the uppermost simulation compartment; in the lowermost compartment, the second is excluded.

To simulate the precipitation of particulate carbon and oxidized iron species, Fe$^+$ and C precipitate in the model. A parameter $p$, where $0 < p < 1$,
determines the balance between vertical eddy diffusion and precipitation for these chemical species so that
the transport rate is
\begin{equation}
  (1 + p) D \left(X_{i-1} - X_i\right) - (1-p) D \left(X_{i+1} - X_i\right).
\end{equation}
Since $p > 0$, these species tend to move down the water column and accumulate above the sediment. As with other species, the first
term in excluded in the top compartment; the second term in the bottom compartment.

\subsection{Reactions}
\subsubsection{Biotically-catalyzed reactions: Primary oxidations}
The oxidation of carbon uses a chain of progressively less energetically-favorable terminal
electron acceptors. Here, we follow the formulation laid out by Hunter et
al.\cite{hunterkinetic1998}, equations 3 and 4.

The total rate of carbon degradation in a compartment follows first order kinetics:
\begin{equation}
  R^\m{C} \equiv k^\m{C} \m{C}; \quad \left( \frac{\partial \m{C}}{\partial t}
  \right)_\text{reaction} = -R^\m{C}.
\end{equation}
The fraction of carbon taken up by oxidation on each of the terminal electron acceptors is
determined by the abundance and relative metabolic merit of the electron acceptors. The $j$-th electron acceptor is
consumed at a rate
\begin{equation}
  R_j = \frac{f_j}{e_j} R^\m{C},
\end{equation}
where $e_j$ is the number of electrons neutralized per electron acceptor molecule and $f_j$
is determined by successive applications of the formula
\begin{equation}
  f_j = \left( 1 - \sum_{k=1}^{j-1} f_k \right) \mathrm{max} \left\{ 1,
  \frac{[\mathrm{EA}_j]}{[\mathrm{EA}_{\mathrm{lim},j}]} \right \}
\end{equation}
for $j \in \{ 1 \ldots 4 \}$. If the $j$-th electron acceptor's concentration $[\mathrm{EA}_j]$ is greater than some constant
limiting concentration $[\mathrm{EA}_{\mathrm{lim},j}]$, then that electron acceptor gets
all the remainder of the carbon; otherwise, it gets a fraction of what is left determined by
the ratio of the two concentrations.

The electron acceptors and their $e_j$ are listed in Table \ref{tab:electron_acceptors}. Methanogenesis
corresponds to $j=5$, and gets all remaining carbon so that $f_5 = 1 - \sum_{k=1}^4 f_k$.
All the carbon allocated by $R^\m{C}$ gets used up (i.e., $\sum_{j=1}^5 = 1$), but each electron
acceptors accepts electrons according to a different stoichiometry (i.e., $\sum_{j=1}^5 R_j \ne
R^\m{C}$).

\begin{table}
\centering
\begin{tabular}{ l l l }
\toprule
$j$ &   EA   &   $e_j$   \\
\midrule
1   &   O   & 4  \\
2   &   N$^+$   & 5  \\
3   &   Fe$^+$  & 1  \\
4   &   S$^+$   & 8  \\
5   &   $\varnothing$   & 8  \\
\bottomrule
\end{tabular}
\caption{Electron acceptors in the primary oxidation reactions. $j=5$ corresponds to
methanogenesis.}
\label{tab:electron_acceptors}
\end{table}

\begin{table}
\begin{tabular}{ l l l l }
\toprule
Primary oxidations  & rate \\
\midrule
$\m{C} \rightarrow a \m{N}^- + e \m{e}^-$   & $R^\m{C}$ & primary oxidation half-reaction \\
$\m{O} \to \varnothing$    &   $R_1$   & aerobic heterotrophy \\
$\m{N}^+ \to \varnothing$    &   $R_2$   & denitrification \\
$\m{Fe}^+ \to \m{Fe}^-$ & $R_3$ & iron reduction  \\
$\m{S}^+ \to \m{S}^-$ & $R_4$ & sulfate reduction  \\
$\varnothing \to \m{M}$ & $R_5$ & methanogenesis \\
\\
\midrule
Biotic secondary oxidations & rate constant \\
\midrule
$2 \m{O} + \m{N}^- \rightarrow \m{N}^+$  & $k_1$ & ammonia oxidation   \\
$2 \m{O} + \m{S}^- \rightarrow \m{S}^+$  & $k_2$ & sulfide oxidation   \\
$\m{N}^+ + 5 \m{Fe}^- \rightarrow 5 \m{Fe}^+$    &   $k_3$  & iron oxidation on nitrate  \\
$\m{CH}_4 + 2 \m{O} \rightarrow \varnothing$    &   $k_4$  & methanotrophy on oxygen  \\
$\m{CH}_4 + \m{S}^+ \rightarrow \m{S}^-$    &   $k_5$  & methanotrophy on sulfate  \\
\\
\midrule
Abiotic secondary oxidations    & rate constant \\
\midrule
$\tfrac{1}{4} \m{O} + \m{Fe}^- \rightarrow \m{Fe}^+$ & $k_6$ & iron oxidation  \\
\bottomrule
\end{tabular}
\label{tab:reactions}
\caption{Reactions simulated in the model.}
\end{table}

\subsection{Secondary oxidations}
We model secondary oxidations, the oxidation of compounds other than carbon compounds, using
second-order mass action kinetics as per Hunter et al.\cite{hunterkinetic1998} (Table 4). For the
transformation of substrates $S_1, S_2$ into a product $P$ according to $a_1 S_1 + a_2 S_2
\to b P$, the reaction rate is $r \equiv k [S_1] [S_2]$ and the reaction terms are
\begin{align}
  \left( \pd{[P]}{t} \right)_\text{reaction} &= b r \\
  \left( \pd{[S_i]}{t} \right)_\text{reaction} &= -a_i r \quad \left( i = 1, 2 \right) \\
\end{align}
with rate constant $k$. As per Hunter et al., we do not adjust the rate according to the reaction's stoichiometry.

Primary and second oxidations are listed in Table \ref{tab:reactions}.

\subsection{Source terms}
Interactions between the hypolimnion and the outside world are modeled by simple source terms. Oxygen and carbon are added at the thermocline. Methane can be produced by primary oxidation in the water column, but methanogenesis also proceeds in the sediment, where it is transported upward and consumed by methanotrophy. We model this process by a point source of methane in the sediment.
All methane in our model is consumed before reaching the thermocline, so we omit the mechanics for emission of methane into the metalimnion.

\subsection{Parameterization}
A list of parameters and their values is included somewhere around here. Most parameters
related to the reaction rates are borrowed from Hunter et al.\cite{hunterkinetic1998}. The
parameters related to transport, sink/source terms, and initial concentrations were drawn
from sources as noted and adjusted by hand. 

\begin{table}
\centering
\begin{tabular}{ l l l }
\toprule
parameter   &   value   & source \\
\midrule
General parameters \\
$T$     & 25 yrs  \\
$N$     & 17    \\
\bottomrule
\end{tabular}
\label{tab:parameters}
\caption{Parameter values and sources.}
\end{table}

\section{Implementation}
The model was implemented in Matlab, and the ODE solutions were computed using the command {\tt ode15s} with all chemical species restricted to nonnegative values (command {\tt odeset}).

\section{Sensitivity analysis}
\subsection{Procedure}
We analyzed the sensitivity of the shape of each rate profile to local perturbations in each parameter using the following steps:

\begin{enumerate}
\item Vary each parameter one-at-a-time from its original value $x$ to five values between and including $0.99x$ and $1.01x$.
\item For each value of the varied parameter, simulate the rates profiles. This produces an simulated rate for each metabolism in each compartment.
\item Fit each rate profile with a cubic interpolation (using the {\tt intrep1d} command in the scientific computing package SciPy for the Python programming language). This produces rate profiles that are defined at non-integer depths.
\item Compute three depths to characterize the rate profile: the depth where the rate is maximized and the (up to) two depths where the rate falls to one-half its maximum value. (If the maximum is at the metalimnion or sediment, then there is only one half-maximum depth.)
\item The sensitivity of the position of each of these characteristic depths to the varied parameter is the slope of the best-fit line of depth against parameter value.
\end{enumerate}

\subsection{Results \& Discussion}
The sensitivities are reported in Table ??.

This sensitivity analysis is purely local. It does not treat the possibility that markedly different parameterizations could yield similar results.

\bibliographystyle{plain}
\bibliography{refs}


\end{document}
